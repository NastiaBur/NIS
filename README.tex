\documentclass{article}
\usepackage{graphicx} % Required for inserting images
% \usepackage{sectsty}
% \allsectionsfont{\centering}

\usepackage{indentfirst}
\usepackage[top=2cm, left=1.5cm, right=1.5cm, bottom=2cm]{geometry}

\documentclass{article}
\usepackage[T2A]{fontenc}
\usepackage{hyperref}

\usepackage{amsmath, amssymb}
\usepackage{amsthm}
\usepackage{cancel}
\usepackage{tikz}

\title{Conditional VAR based on real data (1 dataset)}
\author{Анастасия Буркова и Валерия Рублева}


\begin{document}
\maketitle
Нашим проектом являлся подсчёт $CVaR$ для 1 датасета. Перед тем, как приступить к описанию кода, уточним данный термин.

$CVaR(p)$  - это условная стоимость под риском. $p$  - это вероятность риска (или 1 - уровень доверия). Стандартными значениями $p$ являются $0,01$ и $0.05$. Для подсчета $CVaR(p)$ предполагается, что $VaR(p)$ существует.

Также важно уточнить, что мы использовали исторический метод подсчёта, потому что этот метод дает меньше всего ограничений на рассматриваемую выборку и в предосталенных данных распределение доходностей дискретно и взято из уже прошедших лет, то есть до 2022 года включительно.
\section{Нужные библиотеки}
Перед тем, как начать работу с кодом, мы подключили билиотеки: 
\begin{itemize}
    \item $openpyxl$ (для чтения файла формата $xslx$)
    \item $pandas$ (для работы с датасетом)
    \item $numpy$ (для математических выражениях на массивах)
\end{itemize}

\section{Выгрузка данных}

Чтобы воспользоваться датасетом, находящимся в файле формата $excel$, надо изначально выгрузить в программу все нужные данные. Для этого мы использовали 

$$
df = pd.read\_excel('dataset.xlsx', sheet\_name='1')
$$

Следующие строчки в этом блоке кода видоизменяют таблицу загруженную в программу (меняют названия колонок, удаляют пустые клетки и двигают индексы). Это было сделано, чтобы упростить и сделать более понятной дальнейшую работу с таблицей.

\section{Предподсчёт}
Перед тем как подсчитать $CVaR$, с помощью данной таблицы был получен отдельный столбец (объект типа $pd.Series$): отношение $curs$ нынешней даты к предыдущей. Затем этот столбец был отсортирован по возрастанию и проранжирован. В представленном коде этот объект имеет название $ret$.



\section{Функция для подсчётa $CVaR$}
После подсчета $CVaR$ для конкретных данных, в код была добавлена функция, считающая $CVaR$ для любого процента (переменная $p$) и для любого столбца данных (переменная $data$), которые пользователь хочет указать.  

Алгоритм работы функции:
\begin{enumerate}
    \item Подсчитывается количество данных в $ret$ - переменная count
    \item Находится позиция c номером $p$ процентов от $count$ - переменная $pos$
    \item Берётся среднее значение всех чисел с меньшей позицией (сами значения тоже меньше, чем число, стоящее на $pos$ благодаря сортирвоке) - это $CVaR$
\end{enumerate}


\section{Результаты}
 В итоге получилась функция, которая подсчитывает $CVaR$ для нужного датасета и процента. Запустив эту функцию на стандратных значениях вероятности риска, можно получить условную стоимость на риск для любых дискретных распределений доходности в прошлом.

 Запустив функцию на самом встречаемой вероятности риска (0.01), мы получили результат: 
 0.9048620951800836
 
\href{https://colab.research.google.com/drive/1Kw25f9_MQ9-O2qePLgcIBtVutTpWNZVO?usp=sharing}{Здесь можно ознакомиться с кодом и предподсчетами через Google Colaboratory}. 

\end{document}
